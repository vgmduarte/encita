\documentclass[brazilian, fleqn, 10pt]{article}
\usepackage[portuguese]{babel}
\usepackage{fontspec}
\usepackage[compact]{titlesec}
\usepackage{indentfirst}
\usepackage[margin=2cm]{geometry}
\usepackage{physics}
\usepackage{graphicx}
\usepackage{caption}
\usepackage{booktabs, multirow}
\usepackage{siunitx}
\usepackage{upgreek}
\usepackage{fancyhdr}
\usepackage[capitalise]{cleveref}
\usepackage[backend=biber, natbib=true, style=bath, maxcitenames=1]{biblatex}
\defbibheading{}{}
\addbibresource{references.bib}

\setlength{\parindent}{1.8em}
\captionsetup[figure]{singlelinecheck=false, format=hang, justification=raggedright}
\captionsetup[table]{singlelinecheck=false, format=hang, justification=raggedright, skip=0pt}
\setmainfont{Times New Roman}
\titleformat*{\section}{\normalsize\bfseries}
\titleformat*{\subsection}{\normalsize\bfseries}
\titleformat*{\subsubsection}{\normalsize\bfseries}
\makeatletter
\renewcommand*{\@seccntformat}[1]{\csname the#1\endcsname.\hspace{0.1cm}}
\makeatother
\crefformat{table}{Tab. #1}
\Crefformat{table}{Tabela #1}

\renewcommand\sec[1]{%
    \medskip
    \section{#1}
    \medskip
}
\newcommand\ssec[1]{%
    \medskip
    \subsection{#1}
    \medskip
}
\newcommand\sssec[1]{%
    \medskip
    \subsubsection{#1}
    \medskip
}
\newcommand\be[1]{%
    \medskip
    \begin{#1}
}
\newcommand\ee[1]{%
    \end{#1}

    \medskip
}

% MUDAR DATA -----------------------------------
\renewcommand{\headrulewidth}{0pt}
\pagestyle{fancy}
\fancyhf{}
\lhead{\textit{\small Anais do XXVI ENCITA, ITA, 7 de novembro de 2020}}
\rhead{\small ENCITA-nome do primeiro autor}

\fancypagestyle{firstpage}
{
    \fancyhf{}
    \lhead{\small ENCITA-nome do primeiro autor \\ \textit{Anais do XXVI Encontro de Iniciação Científica e Pós-Graduação do
    ITA - XXVI ENCITA / 2020} \\\textit{Instituto Tecnológico de Aeronáutica, São
    José dos Campos, SP, Brasil, 7 de novembro de 2020}}
}

% FONT SIZE CHEAT SHEET ------------------------
% Essa cheat sheet só serve para \documentclass[10pt]{article}
% \tiny 	         5 pt
% \scriptsize 	     7 pt
% \footnotesize 	 8 pt 
% \small 	         9 pt 	
% \normalsize 	    10 pt 	
% \large 	        12 pt 	
% \Large 	        14 pt 	
% \LARGE 	        17 pt 	
% \huge 	        20 pt 	
% \Huge 	        25 pt 	

% Esse documento deve ser compilado utilizando XeLaTeX
% No Overleaf, vá em Menú (canto superior esquerdo) -> Compiler e mude para XeLaTeX no drop-down menu
\begin{document}

\thispagestyle{firstpage}

$ $

\begin{center}
    \Large \textbf{INSTRUÇÕES PARA A FORMATAÇÃO DE TRABALHOS SUBMETIDOS AO XXVI
    ENCITA}
\end{center}

\vspace{14pt}

\noindent \textbf{Nome do primeiro autor}

\noindent \small Nome da instituição e endereço para correspondência \normalsize

% \noindent Bolsista PIBIC-CNPq % descomentar caso se aplique

\noindent \small E-mail \normalsize

\medskip

\noindent \textbf{Nome do segundo autor}

\noindent \small Nome da instituição e endereço para correspondência \normalsize

% \noindent Bolsista PIBIC-CNPq % descomentar caso se aplique

\noindent \small E-mail \normalsize

% Obs.: o orientador também é um dos co-autores

\medskip
\noindent \small
\emph{%
    \textbf{Resumo.}
    O propósito deste template e servir como modelo de um trabalho a ser
    publicado nos Anais do ENCITA. O resumo deve descrever os objetivos, a
    metodologia e as principais conclusões em não mais de 200 palavras. Ele não
    deve conter fórmulas nem deduções matemáticas. Figuras também não devem
    fazer parte do resumo. O texto deve estar na formatação: Times New Roman,
    italic, size 9.
} \normalsize

\medskip
\noindent \small
\emph{%
    Palavras-chave: palavra-chave 1, palavra-chave 2, palavra-chave 3,
    palavra-chave 4, palavra-chave 5. (até 5)
} \normalsize

\sec{Introdução}

% DEFASADO
% Para inscrever-se no ENCITA o aluno deverá entregar um artigo na secretaria de
% Divisão de Pós-Graduação e Pesquisa. Os artigos deverão ser entregues
% primeiramente em uma cópia impressa. Depois de serem aceitos, os alunos deverão
% enviar a secretária da Divisão de Pós-Graduação e Pesquisa a versão final, em
% uma cópia eletrônica, no formato PDF.

Os Anais do ENCITA serão publicados no site da biblioteca do ITA. Para ser
publicado, o trabalho deve obedecer esta instrução, com respeito ao formato do
texto e qualidade das figuras e tabelas.

Código do trabalho na primeira linha. IMPORTANTE: O autor deve inserir o código
do seu trabalho no formato ENCITA-(nome do primeiro autor).

Os manuscritos devem ser submetidos em Português. O trabalho deve ser digitado
em papel tamanho A4, usando fonte Times Roman, tamanho 10, exceto para o código
do trabalho, o título, os nomes dos autores e afiliações, o resumo e as palavras-chave. 

O trabalho é limitado a 12 (doze) páginas, incluindo tabelas e figuras.

Os nomes dos autores e afiliações devem aparecer já na 1ª versão submetida para
revisão. Caso o trabalho seja aceito sem modificações, os autores não precisarão
enviar nova versão.

Na introdução deve constar a justificativa do trabalho, contextualização e breve
revisão da literatura. Os objetivos do trabalho também devem estar no final da
introdução.

\sec{Título das seções}

O corpo de texto segue a mesma formatação da introdução. Sugere-se que as seções
sejam: 1. Introdução; 2. Material e Métodos; 3. Resultados e Discussão; 4.
Conclusões e Recomendações; 5. Agradecimentos; 6. Referências.

\ssec{Título das sub-seções}

Os títulos e subtítulos das seções devem ser digitados com fonte Times Roman,
tamanho 10, estilo negrito, e alinhados à esquerda. Eles devem ser numerados,
usando numerais arábicos separados por pontos, até o máximo de 3 subníveis. Uma
linha em branco de espaçamento simples deve ser incluída acima e abaixo de cada
título/subtítulo.

\ssec{Corpo do texto}

O corpo do texto deve ser justificado. A primeira linha de cada parágrafo tem
recuo de 10 espaços contados a partir da margem esquerda.

As equações matemáticas devem ser alinhadas à esquerda e citadas como \cref{eq:firsteq} no
meio da frase, ou por \Cref{eq:firsteq} no início de uma frase. Os números das
equações são numerais arábicos colocados entre parênteses, e alinhados à
direita, como mostrado na \cref{eq:firsteq}.

Os símbolos usados nas equações devem ser definidos imediatamente antes ou
depois de sua primeira ocorrência no texto do trabalho.

\newpage

O tamanho da fonte usado nas equações deve ser compatível com o utilizado no
texto. Todos os símbolos devem ter suas unidades expressas no S.I. (Sistema
Internacional).
\be{equation} \label{eq:firsteq}
    \frac{d C}{d w} = \frac{d u}{d w} \cdot \vb{F}_u + \frac{d v}{d w} \cdot \vb{F}_v
\ee{equation}

As figuras devem ser centralizadas e referenciadas como \cref{fig:firstfig} no meio da frase
ou por \Cref{fig:firstfig}, caso apareçam no início. As anotações e numerações devem ter
tamanhos compatíveis com o da fonte usada no texto, e todas as unidades devem
ser expressas no S.I. (Sistema Internacional). Cada figura deve ser colocada na
posição mais próxima possível de sua primeira citação no texto. Deixe uma linha
em branco entre as figuras e o texto. As legendas das figuras devem ser
alinhadas à esquerda.

\begin{figure}[!htbp]
    \medskip
    \centering
    \includegraphics[width=0.3\textwidth]{example-grid-100x100pt}
    \caption{Legenda da figura.}
    \label{fig:firstfig}
\end{figure}

Figuras coloridas e fotografias de alta qualidade podem ser incluídas no
trabalho. É recomendável que qualquer figura inserida no trabalho esteja no
formato GIFF ou JPG.

As tabelas devem ser centralizadas e referidas por \cref{tab:flex} no meio da frase, ou
por \Cref{tab:flex} no início de uma sentença. Os títulos das tabelas devem ser
localizados imediatamente acima da tabela. Anotações e valores numéricos nela
incluídos devem ter tamanhos compatíveis com o da fonte usada no texto do
trabalho, e todas as unidades devem ser expressas no S.I. (Sistema
Internacional). As unidades são incluídas apenas na primeira linha/coluna,
conforme for apropriado. As tabelas devem ser colocadas tão perto, o quanto
possível, de sua primeira citação no texto. Deixe uma linha simples em branco
entre a tabela, seu título e o texto. O estilo de borda da tabela é livre.
Exemplos são apresentados na \cref{tab:flex} e na \cref{tab:proc}.

\begin{table}[!htp]
    \centering
    \caption{Resultados experimentais para as
    propriedades de flexão dos materiais MAT1 and MAT2. Valores médios de
    obtidos em 20 ensaios.}
    \label{tab:flex}
    \scriptsize
    \begin{tabular}{cccc}\toprule
    Propriedades do Compósito &CFRC-TWILL &CFRC-4HS \\\midrule
    Resistência à Flexão (MPa) &$209 \pm 10$ &$180 \pm 15$ \\
    Módulo de Flexão (GPa) &$57 \pm 3$ &$18 \pm 1$ \\
    \bottomrule
    \end{tabular}
\end{table}

As legendas das figuras e das tabelas não devem exceder 3 linhas. A segunda e a
terceira linhas devem ter recuos, como mostrado na legenda da \cref{tab:flex}.

\begin{table}[!htp]\centering
    \caption{Propriedades após o processamento.}
    \label{tab:proc}
    \scriptsize
    \begin{tabular}{cccc}\toprule
    Tipo de processamento &Propriedade 1 &Propriedade 2 \\
    &(\%) &(\si{\upmu m}) \\\midrule
    Processo 1 &40.0 &22.7 \\
    Processo 2 &48.4 &13.9 \\
    Processo 3 &39.0 &22.5 \\
    \bottomrule
    \end{tabular}
\end{table}

A citação das referências no corpo do texto pode ser feita nos formatos:
Bordallo et al. (1989) mostra que o corpo..., ou: Observa-se uma variação linear
da pressão na periferia do rotor, como mostram (Coimbra, 1978; Clark, 1986 e
Sparrow, 1980), que os valores de rigidez da viga... 

Referências aceitáveis incluem: artigos de periódicos, dissertações, teses,
artigos publicados em anais de congressos, ``preprints'' de congressos, livros,
artigos submetidos e aceitos (identificar a fonte), comunicações privadas.

\sec{Agradecimentos}

Esta seção deve ser colocada antes da lista de referências. Não esquecer do
agradecimento ao CNPq.

\sec{Lista de referências}

A lista de referências constitui a última seção do trabalho, sendo denominada
“Referências”.

A primeira linha de cada referência é alinhada à esquerda; todas as outras
linhas têm recuo de 10 espaços da margem esquerda. Todas as referências
incluídas na lista devem aparecer como citações no texto do trabalho.

As referências devem ser postas em ordem alfabética, usando o último nome do
primeiro autor, seguida do ano da publicação. Exemplo da lista de referências é
apresentado a seguir:

\citep{Griffiths2004,Bickelhaupt2000,Gell-Mann1957}

\printbibliography[title={5. Referências}]

\end{document}